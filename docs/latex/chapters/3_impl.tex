\chapter{Fejlesztői dokumentáció} % Developer guide
\label{ch:impl}

\section{TODO}
\subsection{Tartalom}
\begin{itemize}
    \item probléma részletes specifikációja
	\item felhasznált módszerek részletes leírása, a használt fogalmak definíciója
	\item program logikai és fizikai szerkezetének leírása (adatszerkezetek, adatbázisok, modulfelbontás)
	\item bemutatni a program rétegeit, adatbázist, táblákat, osztályokat, modulokat, fontosabb függvényeket, algoritmusokat, felhasználói eseteket, hálózati kommunikációt, fejlesztői környezetet
	\item osztálydiagramok, UML ábrák
	\item üzemeltetésről információk
	\item milyen időzített folyamatok vannak, hol van a logolás, használ-e felhő rendszereket, API-kat
	\item tesztelési terv és a tesztelés eredményei (egység tesztek, felületi tesztek, integrációs tesztek)
	\item ha nem lehet automatikus tesztet írni a programra, akkor írj teszt jegyzőkönyvet (milyen funciót tesztelt, mi a bemenet, mi az elvárt kimenet, mi a tényleges kimenet)
\end{itemize}

\subsection{Bírálati szempontok}
\subsection{Megoldási terv}
Ez a fejlesztői leírás része, a rendszerterv, amelyből az alkalmazás célja, felépítése és működése megérthető, ez alapján az alkalmazás forráskódja lényegében elkészíthető.
    Tartalmazza a következő elemeket:
\begin{itemize}
    \item Rendszer architektúrájának leírását (alrendszerek, rétegek bemutatása, az alkalmazott szabványok, technológiák, fejlesztő módszerek megadása, felhasznált eszközök és kész komponensek definiálása). Az értékelésnél vegyük figyelembe, hogy mennyire válnak szét az alkalmazás rétegei (például felhasználói felület, logika, adatforrás)?
    \item Az adatbázis – feltéve, hogy van – leírását. Érdemes egy áttekintő diagammal szemléltetni a táblákat és a köztük levő kapcsolatokat, majd külön táblázatokban megadni az egyes táblák mezőszerkezeti leírását, az esetleges tárolt eljárások, függvények, triggerek, stb leírását.
    \item Modul és/vagy osztályszerkezet (fontosabb modulok és/vagy osztályok és azok metódusai, továbbá ezek kapcsolatának) leírását. Az egyes csomagok fő eljárásait illetve a fontos osztályok fő metódusait bemenő-, kimenőadat, tevékenység hármassal jellemezni kell.
    \item A felhasználói felület – feltéve, hogy van –  tervét (a képernyő- és listaterveket, valamint a menütervet). Legyen egy áttekintő ábra, amely mutatja a képernyők (ablakok, weblapok) közti navigálási lehetőségeket, irányokat. Ki kell emelni a fontosabb felhasználói eseménykezeléseket.
\end{itemize}

\subsection{Megvalósítás}
A fejlesztői leírásnak a megvalósításról szóló része bemutatja, hogy milyen döntéseket kellett hozni a terv megvalósítása során (adatábrázolás, felhasznált komponensek, kódban alkalmazott nyelvi elemek, stb). A dokumentáció ne tartalmazza a forrásprogramot (legfeljebb csak fontosnak ítélt részleteit), elég azt a mellékelt adathordozón elhelyezni. A megvalósítás a fentieken kívül tartalmazza a komponens tervet (az alkalmazás fizikai komponenseinek kapcsolatrendszerét) és azok telepítésének módját.
    Az értékelésnél vegyük figyelembe:
\begin{itemize}
    \item A forráskód tartalma, szerkezete megfelel-e a tervnek?
    \item Mennyire ismeri a hallgató az adott fejlesztő eszközt (pl. korszerű, hatékony nyelvi elemek vannak-e túlsúlyban, vagy ehelyett bonyolult, nehézkes, körülményes és leginkább terjengős forráskódot eredményező nyelvi elemek jellemzik a kódot)? Indokoltak-e a választott nyelvi elemek használata?
    \item Milyen a forráskód külalakja, mennyire áttekinthető (strukturáltság, bekezdések, tagolások, kommentezés stb.)?
    \item Mennyire módosítható a kód. Alkalmazza-e a hallgató a kód-újrafelhasználás nyelvi eszközeit (függvények, származtatás, generikus elemek)?
    \item Törekszik-e a hatékony adatábrázolásra?
    \item Mennyire öndokumentáló a kód, vagyis a választott azonosítók (pl. változónevek) mennyire beszédesek, konvencionálisak, a megjegyzések mennyire segítik a kódértést?
    \item Tartalmazza a szükséges ellenőrzési, hibakezelési funkciókat, általában megoldott-e a kivételkezelés?
    \item Mennyire gazdálkodik jól az emberi és gépi erőforrásokkal, így például a felhasználó idejével és türelmével, a lemezkapacitással és a memóriakapacitással?
\end{itemize}

\subsection{Tesztelés}
Ez is a fejlesztői leírás része, amelynek a tesztelési szempontokat kell bemutatnia, és a tesztelés során szerzett tapasztalatokat összegeznie valamint a szoftver skálázhatóságáról készített elemzést kell tartalmaznia.
    Az értékelésnél vegyük figyelembe, hogy a dokumentáció:
\begin{itemize}
    \item Tartalmaz-e tesztelési terveket, teszteseteket (Ezeket csoportosíthatja rendszerteszt és modultesztek szerint illetve fekete és fehérdoboz megközelítéssel)?
    \item Beszámol-e olyan tanulságokról, amelyek alapján meg kellett változtatni a korábbi implementációs döntéseket, esetleg a terv egyes elemeit (az ilyen tapasztalatok nem rontják a dolgozat értékét)?
    \item Tartalmazza-e nagy adattömeg melletti futtatások értékelését?
    \item Elemzi-e a program által adott eredmény helyességét (különösen olyan optimalizációs feladatok esetén, ahol több helyes megoldást valamilyen célfüggvénnyel lehet rangsorolni)?
    \item Elemzi-e a program futásának hatékonyságát?

\end{itemize}

\noindent\makebox[\linewidth]{\rule{\paperwidth}{0.4pt}}
% -------------- ITT KEZDŐDIK A LÉNYEGI RÉSZ ------------------------
