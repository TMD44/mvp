\chapter{Felhasználói dokumentáció} % User guide
\label{ch:user}

\section{TODO}
\subsection{Tartalom}
\begin{itemize}
    \item megoldott probléma rövid megfogalmazása
	\item felhasznált módszerek rövid leírása
	\item program használatához szükséges összes információ, gépigény, telepítés, futtatás
	\item alkalmazás bemutatása hogy az átlag felhasználó megértse
	\item képernyőképek amik segítik a program használatát
	\item use-case-ek használata amivel bemutatom a funkciókat
\end{itemize}

\subsection{Bírálati szempontok}
Magába foglalja a telepítési- (vagy üzemeltetési-) és a végfelhasználói leírást. Ezek meghatározott célközönséghez szólnak, könnyen és gyorsan kell, hogy eligazítsák a felhasználót a program használatában!
Tartalma:
\begin{itemize}
    \item A feladat rövid ismertetése (mire való a szoftver)
	\item Célközönség (kik, mikor, mire használhatják a programot)
	\item A rendszer használatához szükséges minimális, illetve optimális HW/SW környezet
    \item Első üzembe helyezés leírása – ha van ilyen –, a program indítása (kivéve, ha nem egy önálló alkalmazásról, hanem egy meglévő rendszer új komponenséről van szó). Itt ellenőrizzük, hogy a telepítési útmutató megfelel-e a valóságos telepítési folyamatnak.
    \item Általános felhasználói tájékoztató (például a szokásostól eltérő képernyő-, billentyű-, illetve egérkezelés leírása, teendők hibaüzenetek esetén stb.).
    \item A rendszer funkcióinak ismertetése. A feladat jellegéből fakadóan célszerű lehet ezt folyamatszerűen, képernyőképekkel alátámasztva bemutatni. A funkciókat ajánlatos a felhasználói szintek szerint csoportosítani. Itt vegyük figyelembe, hogy a leírás a fejlesztői dokumentációban meghatározott részfeladathoz illeszkedik-e, az ott meghatározott funkciókat/használati eseteket írja-e le?
    \item A rendszer futás közbeni üzenetei (hibaüzenetek, figyelmeztető üzenetek, felszólító üze-netek stb.) és azok magyarázata – az esetleges üzemeltetési teendőkkel együtt. Itt vegyük figyelembe, hogy tartalmaz-e biztonsági, illetve hibaelhárítási előírásokat?
    \item Egyéb, a szoftver használatához szükséges információk.

\end{itemize}

\noindent\makebox[\linewidth]{\rule{\paperwidth}{0.4pt}}
% -------------- ITT KEZDŐDIK A LÉNYEGI RÉSZ ------------------------
