\chapter{Bevezetés} % Introduction
\label{ch:intro}

\section{TODO}
\subsection{Tartalom}
\begin{itemize}
    \item miről is fog szólni a szakdolgozat
	\item témaválasztás indoklása
	\item megoldandó feladat rövid, közérthető leírása, milyen problémát old meg
	\item motivációk
	\item olvasó érdeklődésének felkeltése
\end{itemize}

\subsection{Bírálati szempontok}
\subsection{A megoldott programozási feladat nehézsége}
Ha a feladat éppen teljesíti a szakdolgozatként elvárt nehézségi szintet, akkor az 1 pontot ér. Többlet pontok akkor járnak, ha például:
\begin{itemize}
    \item A feladat megoldása igényelt a kötelező tananyagon kívüli ismereteket.
    \item A megoldás alkalmaz nem magától értetődő implementációjú algoritmusokat.
    \item Szükség volt-e a megoldáshoz különböző platformok vegyes alkalmazására.
    \item Nagy méretű adatbázist kellett létrehozni tárolt eljárásokkal, szerverágensek által kezelt elemekkel, kapcsolatot teremteni különféle egyéb szerverekkel (pl. Notification Server).
    \item A szoftver egy nagyobb rendszer komponense, ahol a környezetet is meg kellett ismerni a fejlesztés előtt.
\end{itemize}

\subsection{A dolgozat felépítése, nyelvezete és külalakja}
A hallgatónak be kell tartania az általános követelményekben megadott szempontokat. Ezen túlmenően az értékelésnél vegyük figyelembe:
\begin{itemize}
	\item Tartalmazza-e a diplomamunka a kötelező részeket (témabejelentő, felhasználói- és fejlesztői leírás, irodalomjegyzék, tartalomjegyzék)?
    \item A szöveg szerkezete, fejezetbontása helyénvaló, logikus, érthető-e? Milyen a dolgozat részeinek kapcsolata, összhangja?
    \item Mennyire áttekinthető a szöveg? Alkalmazza a hallgató a szövegszerkesztési technikákat (lapszámozás, címsorok, tartalomjegyzék, stílusok, futó fejléc, ábraszámozás, stb.)
    \item Milyen a dolgozat formája esztétikai, nyelvtani és stiláris szempontból?
    \item Az ábrák és a szövegek aránya megfelelő-e? Olvashatók, helyénvalók az ábrák?
    \item Megfelelően jelöli-e, hivatkozza-e meg a dolgozat a más forrásokból átvett anyagokat?
\end{itemize}

\noindent\makebox[\linewidth]{\rule{\paperwidth}{0.4pt}}
% -------------- ITT KEZDŐDIK A LÉNYEGI RÉSZ ------------------------

A mai világban elég hálás dolog film- és sorozatrajongónak lenni...
