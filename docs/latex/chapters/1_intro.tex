\chapter{Bevezetés} % Introduction
\label{ch:intro}

A mai világban elég hálás dolog film- és sorozatrajongónak lenni, rengeteg tartalommal bombáznak minket a különböző streaming platformok (Netflix, HBO GO), TV csatornák, Youtube csatornák és még sorolhatnánk. Mindazonáltal természetesen semmi sem tökéletes: biztos a Kedves Olvasó is tapasztalta már, hogy kereste a kedvenc filmjét, sorozatát az éppen akutálisan előfizetett platformon vagy TV-ben. Azoban szinte törvényszerű, hogy amit éppen akarunk nézni az sosincs fent a kínálatban, vagy még rosszabb, amikor fent van, de nem olyan formában, ahogy nekünk az megfelelő (például nincs magyar vagy angol szinkron, magyar felirat stb.), ez tipikus esete a ``so close, yet so far'' szituációnak.

Mindeközben polcainkon egyre porosodnak - vagy már réges-régen lekerültek onnan - a nem használt CD-k, DVD-k, Blu-Ray lemezek és így 2021-ben élérkeztünk oda is, hogy a nagy becsben tartott kedvenc filmjeinket tartalmazó több terrabájtos külső merevlemezünk is a por és a feledés martalékává válik.

Témaválasztásomat és motivációmat, tehát ezen téma ihlette meg leginkább, hiszen hasonló cipőben járok jómagam is. Temérdek kedvenc filmemnek szerettem volna egy átlátható és szép környezetet biztosítani, amelyen ezen tartalmak fogyasztása mégnagyobb élmény.

Alkalmazásommal ezekre a problémákra szeretnék megoldást nyújtani: egy offline tartalmakra, tehát külső meghajtón, SSD-n vagy pendrive-on tárolt és ezekre a tartalmakra épülő, megjelenésében és funckióiban a már megszokott streaming szolgáltatókra hasonlító, könnyen kezelhető és áttekinthető platform létrehozása.\\
Ez a Multimédia Vizualizációs Platform vagy röviden MVP.

Funkcióit tekintve biztosít mindent amit egy videó lejátszó platformtól el lehet várni:
\begin{itemize}
    \item Elemek listázása
    \item Keresés
	\item Filmek,sorozatok lejátszása, megállítása
	\item Előre-hátra pörgetés
	\item Felirat választás
	\item Lejátszási listák létrehozása
	\item Külön kezeli a filmek és a többrészes sorozatok felületeit
	\item Metaadatok betöltése fájlnévből és NFO fájlból
	\item Metaadatok letöltése adatbázisból
\end{itemize}

