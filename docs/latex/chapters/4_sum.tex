\chapter{Összegzés} % Conclusion
\label{ch:sum}

\section{TODO}
\subsection{Tartalom}
\begin{itemize}
    \item egy oldalas összefoglalás
	\item újra bemutatni a problémát amivel foglalkoztam, mit, mire, hogyan használtam
	\item végül bemutatni hogy mire jutottam
\end{itemize}

\subsection{Bírálati szempontok}
\subsection{Szoftver futtatása}
\subsection{Helyesség:}
\begin{itemize}
    \item A feladat-meghatározásnak, illetve a tervnek megfelelően működik-e a program?
\end{itemize}
\subsection{Robusztusság:}
\begin{itemize}
    \item Mennyire védett a felhasználói hibákkal szemben a program (elronthatatlan/hibatűrő)?
    \item Életszerű (nagy mennyiségű) adatok esetén is hatékony munkavégzést biztosít-e a felhasználó számára a program?
    \item Ha a rendszer „jóindulatú” tesztelés esetén elszáll, akkor a munka nem fogadható el.
\end{itemize}
\subsection{Felhasználó-barátság (szabványos felület, kényelem):}
\begin{itemize}
    \item A program könnyen használható? Áttekinthető? Rugalmas? Esztétikus?
    \item A felhasználói felület támogatja a szakterületi feladat elvégzését?
    \item A feliratok konvencionálisak? Egyértelműek a megfogalmazások? Jó a helyesírás?
    \item Van-e segítség (helyzet-érzékeny súgó)?
    \item Indokolt esetben a műveletek megszakíthatók-e, illetve védettek-e megszakítás ellen?
    \item Ismeri, illetve figyelembe veszi a felhasználói tradíciókat?
\end{itemize}


\section{PONTOZÁS}
\begin{enumerate}
	\item\label{step:first} A megoldott programozási feladat nehézsége {\bf 4 PONT}
	\item A dolgozat felépítése, nyelvezete, külalakja {\bf 4 PONT}
    \item Felhasználói dokumentáció {\bf 4 PONT}
    \item Megoldási terv (fejlesztői dokumentáció) {\bf 4 PONT}
    \item Megvalósítás (fejlesztői dokumentáció) {\bf 4 PONT}
    \item Tesztelés (fejlesztői dokumentáció) {\bf 4 PONT}
    \item Szoftver futtatása {\bf 6 PONT}
\end{enumerate}

{\bf ÖSSZESEN: 30 PONT}

\begin{itemize}
    \item  0-14: elégtelen
    \item 15-18: elégséges
    \item 19-22: közepes
    \item 23-26: jó
    \item 27-30: jeles
\end{itemize}

\noindent\makebox[\linewidth]{\rule{\paperwidth}{0.4pt}}
% -------------- ITT KEZDŐDIK A LÉNYEGI RÉSZ ------------------------
