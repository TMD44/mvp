\documentclass[
	%parspace, % Térköz bekezdések közé / Add vertical space between paragraphs
	%noindent, % Bekezdésének első sora ne legyen behúzva / No indentation of first lines in each paragraph
	%nohyp, % Szavak sorvégi elválasztásának tiltása / No hyphenation of words
	%twoside, % Kétoldalas nyomtatás / Double sided format
	%draft, % Gyorsabb fordítás ábrák rajzolása nélkül / Quicker draft compilation without rendering images
	%final, % Teendők elrejtése / Set final to hide todos
]{elteikthesis}[2020/11/23]

% Dolgozat metaadatai
% Document's metadata
\title{Multimédia Vizualizációs Platform} % cím / title
\date{2021} % védés éve / year of defense

% Szerző metaadatai
% Author's metadata
\author{Tarjáni Martin Dominik}
\degree{programtervező informatikus BSc}

% Témavezető(k) metaadatai
% Superivsor(s)' metadata
\supervisor{Tarcsi Ádám} % belső témavezető neve / internal supervisor's name
\affiliation{egyetemi tanársegéd} % belső témavezető beosztása / internal supervisor's affiliation
%\extsupervisor{Külső Kornél} % külső témavezető neve / external supervisor's name
%\extaffiliation{informatikai igazgató} % külső témavezető beosztása / external supervisor's affiliation

% Egyetem metaadatai
% University's metadata
\university{Eötvös Loránd Tudományegyetem} % egyetem neve / university's name
\faculty{Informatikai Kar} % kar neve / faculty's name
\department{Adattudományi és Adattechnológiai \\ Tanszék} % tanszék neve / department's name
\city{Budapest} % város / city
\logo{elte_cimer_szines} % logo

% Témabejelentő metaadatai
% Thesis Proposal metadata
\proposaltitle{Szakdolgozati témabejelentő}
\proposal{A Multimédia Vizualizációs Platform egy számítógépre, külső HDD-re, SSD-re vagy Pendrive-ra letöltött multimédia - főleg filmek, sorozatok - vizualizációjára szolgál, tehát ezen tartalmak fogyasztására egy áttekinthető és könnyen kezelhető felületet biztosít. Hasonlót, amelyhez már hozzászokhattunk a különböző streaming szolgáltatóktól, csak mindezt nem egy külső adatbázisból betöltve, hanem a saját offline tartalmainkra építve. A felület biztosítja az alapvető funkcionalitásokat amelyeket egy videó lejátszó platformtól el lehet várni: filmek listázása, keresés, lejátszás, megállítás, előre-hátra pörgetés, nyelv és felirat választás, lejátszási listák létrehozása stb. Külön kezeli az egyrészes filmek, a több részes filmsorozatok és a több részes TV sorozatok felületeit. A filmekhez és sorozatokhoz továbbá meta-adatokat (Cím, Szereplők, Leírás, Műfajok stb.) tölt le/be, ezzel is növelve a felhasználói élményt.}
\cityproposal{Budapest}
\dateproposal{2020.12.01.}

% Köszönetnyilvánítás metaadatai
% Acknowledgments matadata
\acknowledgmentstext{Lorem ipsum dolor sit amet, consectetur adipiscing elit. Pellentesque facilisis in nibh auctor molestie. Donec porta tortor mauris. Cras in lacus in purus ultricies blandit. Proin dolor erat, pulvinar posuere orci ac, eleifend ultrices libero. Donec elementum et elit a ullamcorper. Nunc tincidunt, lorem et consectetur tincidunt, ante sapien scelerisque neque, eu bibendum felis augue non est. Maecenas nibh arcu, ultrices et libero id, egestas tempus mauris. Etiam iaculis dui nec augue venenatis, fermentum posuere justo congue. Nullam sit amet porttitor sem, at porttitor augue. Proin bibendum justo at ornare efficitur. Donec tempor turpis ligula, vitae viverra felis finibus eu. Curabitur sed libero ac urna condimentum gravida. Donec tincidunt neque sit amet neque luctus auctor vel eget tortor. Integer dignissim, urna ut lobortis volutpat, justo nunc convallis diam, sit amet vulputate erat eros eu velit. Mauris porttitor dictum ante, commodo facilisis ex suscipit sed.}

% Irodalomjegyzék hozzáadása
% Add bibliography file
\addbibresource{thesis.bib}

% A dolgozat
% The document
\begin{document}

% Nyelv kiválasztása
% Set document language
\documentlang{magyar}
%\documentlang{english}

% Teendők listája (final dokumentumban nincs)
% List of todos (not in the final document)
%\listoftodos[\todolabel]

% Dokumentum beállítások
% Some document settings
\input{settings.tex}

% Címlap (kötelező)
% Title page (mandatory)
\maketitle
%\topicdeclaration

% Témabejelentő (kötelező)
% Thesis Proposal (mandatory)
\thesisproposal

% Köszönetnyilvánítás
% Acknowledgments
\acknowledgments

% Tartalomjegyzék (kötelező)
% Table of contents (mandatory)
\tableofcontents
\cleardoublepage

% Tartalom
% Main content
\input{chapters/intro.tex}
\cleardoublepage

\input{chapters/user.tex}
\cleardoublepage

\input{chapters/impl.tex}
\cleardoublepage

\input{chapters/sum.tex}
\cleardoublepage

% Függelékek (opcionális) - hosszabb részletező táblázatok, sok és/vagy nagy kép esetén hasznos
% Appendices (optional) - useful for detailed information in long tables, many and/or large figures, etc.
\appendix
\chapter{Továbbfejlesztési lehetőségek}
\label{appx:further_development}

Az alkalmazás természetesen még közel se tökéletes, rengeteg lehetőség áll még előtte.

\cleardoublepage

% Irodalomjegyzék (kötelező)
% Bibliography (mandatory)
\addcontentsline{toc}{chapter}{\biblabel}
\printbibliography[title=\biblabel]
\cleardoublepage

% Ábrajegyzék (opcionális) - 3-5 ábra fölött érdemes
% List of figures (optional) - useful over 3-5 figures
\addcontentsline{toc}{chapter}{\lstfigurelabel}
\listoffigures
\cleardoublepage

% Táblázatjegyzék (opcionális) - 3-5 táblázat fölött érdemes
% List of tables (optional) - useful over 3-5 tables
\addcontentsline{toc}{chapter}{\lsttablelabel}
\listoftables
\cleardoublepage

% Forráskódjegyzék (opcionális) - 3-5 kódpélda fölött érdemes
% List of codes (optional) - useful over 3-5 code samples
\addcontentsline{toc}{chapter}{\lstcodelabel}
\lstlistoflistings
\cleardoublepage

% Jelölésjegyzék (opcionális)
% List of symbols (optional)
%\printnomenclature

\end{document}
