\documentclass[
	%parspace, % Térköz bekezdések közé / Add vertical space between paragraphs
	%noindent, % Bekezdésének első sora ne legyen behúzva / No indentation of first lines in each paragraph
	%nohyp, % Szavak sorvégi elválasztásának tiltása / No hyphenation of words
	%twoside, % Kétoldalas nyomtatás / Double sided format
	%draft, % Gyorsabb fordítás ábrák rajzolása nélkül / Quicker draft compilation without rendering images
	%final, % Teendők elrejtése / Set final to hide todos
]{elteikthesis}[2020/11/23]

% Dolgozat metaadatai
% Document's metadata
\title{Multimédia Vizualizációs Platform} % cím / title
\date{2021} % védés éve / year of defense

% Szerző metaadatai
% Author's metadata
\author{Tarjáni Martin Dominik}
\degree{programtervező informatikus BSc}

% Témavezető(k) metaadatai
% Superivsor(s)' metadata
\supervisor{Tarcsi Ádám} % belső témavezető neve / internal supervisor's name
\affiliation{egyetemi tanársegéd} % belső témavezető beosztása / internal supervisor's affiliation
%\extsupervisor{Külső Kornél} % külső témavezető neve / external supervisor's name
%\extaffiliation{informatikai igazgató} % külső témavezető beosztása / external supervisor's affiliation

% Egyetem metaadatai
% University's metadata
\university{Eötvös Loránd Tudományegyetem} % egyetem neve / university's name
\faculty{Informatikai Kar} % kar neve / faculty's name
\department{Adattudományi és Adattechnológiai \\ Tanszék} % tanszék neve / department's name
\city{Budapest} % város / city
\logo{elte_cimer_szines} % logo

% Témabejelentő metaadatai
% Thesis Proposal metadata
\proposaltitle{Szakdolgozati témabejelentő}
\proposal{A Multimédia Vizualizációs Platform egy számítógépre, külső HDD-re, SSD-re vagy Pendrive-ra letöltött multimédia - főleg filmek, sorozatok - vizualizációjára szolgál, tehát ezen tartalmak fogyasztására egy áttekinthető és könnyen kezelhető felületet biztosít. Hasonlót, amelyhez már hozzászokhattunk a különböző streaming szolgáltatóktól, csak mindezt nem egy külső adatbázisból betöltve, hanem a saját offline tartalmainkra építve. A felület biztosítja az alapvető funkcionalitásokat amelyeket egy videó lejátszó platformtól el lehet várni: filmek listázása, keresés, lejátszás, megállítás, előre-hátra pörgetés, nyelv és felirat választás, lejátszási listák létrehozása stb. Külön kezeli az egyrészes filmek, a több részes filmsorozatok és a több részes TV sorozatok felületeit. A filmekhez és sorozatokhoz továbbá meta-adatokat (Cím, Szereplők, Leírás, Műfajok stb.) tölt le/be, ezzel is növelve a felhasználói élményt.}
\cityproposal{Budapest}
\dateproposal{2020.12.01.}

% Köszönetnyilvánítás metaadatai
% Acknowledgments matadata
\acknowledgmentstext{Lorem ipsum dolor sit amet, consectetur adipiscing elit. Pellentesque facilisis in nibh auctor molestie. Donec porta tortor mauris. Cras in lacus in purus ultricies blandit. Proin dolor erat, pulvinar posuere orci ac, eleifend ultrices libero. Donec elementum et elit a ullamcorper. Nunc tincidunt, lorem et consectetur tincidunt, ante sapien scelerisque neque, eu bibendum felis augue non est. Maecenas nibh arcu, ultrices et libero id, egestas tempus mauris. Etiam iaculis dui nec augue venenatis, fermentum posuere justo congue. Nullam sit amet porttitor sem, at porttitor augue. Proin bibendum justo at ornare efficitur. Donec tempor turpis ligula, vitae viverra felis finibus eu. Curabitur sed libero ac urna condimentum gravida. Donec tincidunt neque sit amet neque luctus auctor vel eget tortor. Integer dignissim, urna ut lobortis volutpat, justo nunc convallis diam, sit amet vulputate erat eros eu velit. Mauris porttitor dictum ante, commodo facilisis ex suscipit sed.}

% Irodalomjegyzék hozzáadása
% Add bibliography file
\addbibresource{thesis.bib}

% A dolgozat
% The document
\begin{document}

% Nyelv kiválasztása
% Set document language
\documentlang{magyar}
%\documentlang{english}

% Teendők listája (final dokumentumban nincs)
% List of todos (not in the final document)
%\listoftodos[\todolabel]

% Dokumentum beállítások
% Some document settings
% Lábjegyzet folytonos számozása fejezetek között
% Continuous counting of footnotes among chapters
\counterwithout{footnote}{chapter}

% Tartalomjegyzék oldalszámozásának rejtése
% Hide page numbering of ToC
\newcounter{conpageno}
\let\oldtableofcontents\tableofcontents
\renewcommand{\tableofcontents}{
	\pagenumbering{gobble}
	\oldtableofcontents
	\cleardoublepage
	\setcounter{conpageno}{\value{page}}
	\pagenumbering{arabic}
	\setcounter{page}{\value{conpageno}}
}


% Címlap (kötelező)
% Title page (mandatory)
\maketitle
%\topicdeclaration

% Témabejelentő (kötelező)
% Thesis Proposal (mandatory)
\thesisproposal

% Köszönetnyilvánítás
% Acknowledgments
\acknowledgments

% Tartalomjegyzék (kötelező)
% Table of contents (mandatory)
\tableofcontents
\cleardoublepage

% Tartalom
% Main content
\chapter{Bevezetés} % Introduction
\label{ch:intro}

A mai világban elég hálás dolog film- és sorozat rajongónak lenni, rengeteg tartalommal bombáznak minket a különböző streaming platformok (Netflix, HBO GO), TV csatornák, Youtube csatornák és még sorolhatnánk. Mindazonáltal természetesen semmi sem tökéletes: biztos a Kedves Olvasó is tapasztalta már, hogy kereste a kedvenc filmjét, sorozatát az éppen aktuálisan előfizetett platformon vagy TV-ben. Azonban szinte törvényszerű, hogy amit éppen akarunk nézni az sosincs fent a kínálatban, vagy még rosszabb, amikor fent van, de nem olyan formában, ahogy nekünk az megfelelő (például nincs magyar vagy angol szinkron, magyar felirat stb.).

Mindeközben polcainkon egyre porosodnak - vagy már réges-régen lekerültek onnan - a nem használt CD, DVD és Blu-Ray lemezek és így 2021-ben elérkeztünk oda is, hogy a nagy becsben tartott kedvenc filmjeinket tartalmazó több terrabájtos külső merevlemezünk is a por és a feledés martalékává válik.

Téma választásomat és motivációmat, tehát ezen téma ihlette meg leginkább, hiszen hasonló cipőben járok jómagam is. Temérdek kedvenc filmemnek szerettem volna egy átlátható és szép környezetet biztosítani, amelyen ezen tartalmak fogyasztása még nagyobb élmény.

Alkalmazásommal ezekre a problémákra szeretnék megoldást nyújtani: egy offline tartalmakra, tehát külső meghajtón, SSD-n vagy pendrive-on tárolt és ezekre a tartalmakra épülő, megjelenésében és funkcióiban a már megszokott streaming szolgáltatókra hasonlító, könnyen kezelhető és áttekinthető platform létrehozása.\\
Ez a Multimédia Vizualizációs Platform vagy röviden MVP.

Funkcióit tekintve biztosít mindent, amit egy videó lejátszó platformtól el lehet várni:
\begin{itemize}
    \item Elemek listázása
    \item Keresés
	\item Filmek, sorozatok lejátszása, megállítása
	\item Előre-hátra pörgetés
	\item Felirat választás
	\item Lejátszási listák létrehozása
	\item Külön kezeli a filmek és a több részes sorozatok felületeit
	\item Metaadatok betöltése fájlnévből és NFO fájlból
	\item Metaadatok letöltése TMDB adatbázisból
\end{itemize}


\cleardoublepage

\chapter{Felhasználói dokumentáció} % User guide
\label{ch:user}

\section{TODO}
\subsection{Tartalom}
\begin{itemize}
    \item megoldott probléma rövid megfogalmazása
	\item felhasznált módszerek rövid leírása
	\item program használatához szükséges összes információ, gépigény, telepítés, futtatás
	\item alkalmazás bemutatása hogy az átlag felhasználó megértse
	\item képernyőképek amik segítik a program használatát
	\item use-case-ek használata amivel bemutatom a funkciókat
\end{itemize}

\subsection{Bírálati szempontok}
Magába foglalja a telepítési- (vagy üzemeltetési-) és a végfelhasználói leírást. Ezek meghatározott célközönséghez szólnak, könnyen és gyorsan kell, hogy eligazítsák a felhasználót a program használatában!
Tartalma:
\begin{itemize}
    \item A feladat rövid ismertetése (mire való a szoftver)
	\item Célközönség (kik, mikor, mire használhatják a programot)
	\item A rendszer használatához szükséges minimális, illetve optimális HW/SW környezet
    \item Első üzembe helyezés leírása – ha van ilyen –, a program indítása (kivéve, ha nem egy önálló alkalmazásról, hanem egy meglévő rendszer új komponenséről van szó). Itt ellenőrizzük, hogy a telepítési útmutató megfelel-e a valóságos telepítési folyamatnak.
    \item Általános felhasználói tájékoztató (például a szokásostól eltérő képernyő-, billentyű-, illetve egérkezelés leírása, teendők hibaüzenetek esetén stb.).
    \item A rendszer funkcióinak ismertetése. A feladat jellegéből fakadóan célszerű lehet ezt folyamatszerűen, képernyőképekkel alátámasztva bemutatni. A funkciókat ajánlatos a felhasználói szintek szerint csoportosítani. Itt vegyük figyelembe, hogy a leírás a fejlesztői dokumentációban meghatározott részfeladathoz illeszkedik-e, az ott meghatározott funkciókat/használati eseteket írja-e le?
    \item A rendszer futás közbeni üzenetei (hibaüzenetek, figyelmeztető üzenetek, felszólító üze-netek stb.) és azok magyarázata – az esetleges üzemeltetési teendőkkel együtt. Itt vegyük figyelembe, hogy tartalmaz-e biztonsági, illetve hibaelhárítási előírásokat?
    \item Egyéb, a szoftver használatához szükséges információk.

\end{itemize}

\noindent\makebox[\linewidth]{\rule{\paperwidth}{0.4pt}}
% -------------- ITT KEZDŐDIK A LÉNYEGI RÉSZ ------------------------

\cleardoublepage

\chapter{Fejlesztői dokumentáció} % Developer guide
\label{ch:impl}

\section{TODO}
\subsection{Tartalom}
\begin{itemize}
    \item probléma részletes specifikációja
	\item felhasznált módszerek részletes leírása, a használt fogalmak definíciója
	\item program logikai és fizikai szerkezetének leírása (adatszerkezetek, adatbázisok, modulfelbontás)
	\item bemutatni a program rétegeit, adatbázist, táblákat, osztályokat, modulokat, fontosabb függvényeket, algoritmusokat, felhasználói eseteket, hálózati kommunikációt, fejlesztői környezetet
	\item osztálydiagramok, UML ábrák
	\item üzemeltetésről információk
	\item milyen időzített folyamatok vannak, hol van a logolás, használ-e felhő rendszereket, API-kat
	\item tesztelési terv és a tesztelés eredményei (egység tesztek, felületi tesztek, integrációs tesztek)
	\item ha nem lehet automatikus tesztet írni a programra, akkor írj teszt jegyzőkönyvet (milyen funciót tesztelt, mi a bemenet, mi az elvárt kimenet, mi a tényleges kimenet)
\end{itemize}

\subsection{Bírálati szempontok}
\subsection{Megoldási terv}
Ez a fejlesztői leírás része, a rendszerterv, amelyből az alkalmazás célja, felépítése és működése megérthető, ez alapján az alkalmazás forráskódja lényegében elkészíthető.
    Tartalmazza a következő elemeket:
\begin{itemize}
    \item Rendszer architektúrájának leírását (alrendszerek, rétegek bemutatása, az alkalmazott szabványok, technológiák, fejlesztő módszerek megadása, felhasznált eszközök és kész komponensek definiálása). Az értékelésnél vegyük figyelembe, hogy mennyire válnak szét az alkalmazás rétegei (például felhasználói felület, logika, adatforrás)?
    \item Az adatbázis – feltéve, hogy van – leírását. Érdemes egy áttekintő diagammal szemléltetni a táblákat és a köztük levő kapcsolatokat, majd külön táblázatokban megadni az egyes táblák mezőszerkezeti leírását, az esetleges tárolt eljárások, függvények, triggerek, stb leírását.
    \item Modul és/vagy osztályszerkezet (fontosabb modulok és/vagy osztályok és azok metódusai, továbbá ezek kapcsolatának) leírását. Az egyes csomagok fő eljárásait illetve a fontos osztályok fő metódusait bemenő-, kimenőadat, tevékenység hármassal jellemezni kell.
    \item A felhasználói felület – feltéve, hogy van –  tervét (a képernyő- és listaterveket, valamint a menütervet). Legyen egy áttekintő ábra, amely mutatja a képernyők (ablakok, weblapok) közti navigálási lehetőségeket, irányokat. Ki kell emelni a fontosabb felhasználói eseménykezeléseket.
\end{itemize}

\subsection{Megvalósítás}
A fejlesztői leírásnak a megvalósításról szóló része bemutatja, hogy milyen döntéseket kellett hozni a terv megvalósítása során (adatábrázolás, felhasznált komponensek, kódban alkalmazott nyelvi elemek, stb). A dokumentáció ne tartalmazza a forrásprogramot (legfeljebb csak fontosnak ítélt részleteit), elég azt a mellékelt adathordozón elhelyezni. A megvalósítás a fentieken kívül tartalmazza a komponens tervet (az alkalmazás fizikai komponenseinek kapcsolatrendszerét) és azok telepítésének módját.
    Az értékelésnél vegyük figyelembe:
\begin{itemize}
    \item A forráskód tartalma, szerkezete megfelel-e a tervnek?
    \item Mennyire ismeri a hallgató az adott fejlesztő eszközt (pl. korszerű, hatékony nyelvi elemek vannak-e túlsúlyban, vagy ehelyett bonyolult, nehézkes, körülményes és leginkább terjengős forráskódot eredményező nyelvi elemek jellemzik a kódot)? Indokoltak-e a választott nyelvi elemek használata?
    \item Milyen a forráskód külalakja, mennyire áttekinthető (strukturáltság, bekezdések, tagolások, kommentezés stb.)?
    \item Mennyire módosítható a kód. Alkalmazza-e a hallgató a kód-újrafelhasználás nyelvi eszközeit (függvények, származtatás, generikus elemek)?
    \item Törekszik-e a hatékony adatábrázolásra?
    \item Mennyire öndokumentáló a kód, vagyis a választott azonosítók (pl. változónevek) mennyire beszédesek, konvencionálisak, a megjegyzések mennyire segítik a kódértést?
    \item Tartalmazza a szükséges ellenőrzési, hibakezelési funkciókat, általában megoldott-e a kivételkezelés?
    \item Mennyire gazdálkodik jól az emberi és gépi erőforrásokkal, így például a felhasználó idejével és türelmével, a lemezkapacitással és a memóriakapacitással?
\end{itemize}

\subsection{Tesztelés}
Ez is a fejlesztői leírás része, amelynek a tesztelési szempontokat kell bemutatnia, és a tesztelés során szerzett tapasztalatokat összegeznie valamint a szoftver skálázhatóságáról készített elemzést kell tartalmaznia.
    Az értékelésnél vegyük figyelembe, hogy a dokumentáció:
\begin{itemize}
    \item Tartalmaz-e tesztelési terveket, teszteseteket (Ezeket csoportosíthatja rendszerteszt és modultesztek szerint illetve fekete és fehérdoboz megközelítéssel)?
    \item Beszámol-e olyan tanulságokról, amelyek alapján meg kellett változtatni a korábbi implementációs döntéseket, esetleg a terv egyes elemeit (az ilyen tapasztalatok nem rontják a dolgozat értékét)?
    \item Tartalmazza-e nagy adattömeg melletti futtatások értékelését?
    \item Elemzi-e a program által adott eredmény helyességét (különösen olyan optimalizációs feladatok esetén, ahol több helyes megoldást valamilyen célfüggvénnyel lehet rangsorolni)?
    \item Elemzi-e a program futásának hatékonyságát?

\end{itemize}

\noindent\makebox[\linewidth]{\rule{\paperwidth}{0.4pt}}
% -------------- ITT KEZDŐDIK A LÉNYEGI RÉSZ ------------------------

\cleardoublepage

\chapter{Összegzés} % Conclusion
\label{ch:sum}

Az alkalmazás tehát filmek és sorozatok vizualizációját tűzte ki célul, egy jól átlátható, egyszerű felületet a média tartalmaink megtekintésére, belső és külső lejátszóban egyaránt, továbbá listázó plaformként nagyobb filmes kollekciókat egy sokkal látványosabb környezetben tudjuk szemlélni, mintha csak simán a Windows Fájlkezelőben böngészgetnénk, köszönhető ez a tartalmainkhoz metaadatokat be- és letöltő funkciónak is.

A program megalkotása során rengeteget tanultam és tapasztaltam, a számomra eddig idegen és elég kaotikus Node-TypeScript-Electron-React világot már sokkal közelebbinek érzem és átlátom annak működését. Nagyobb projekt megvalósításaként olyan készségeket szereztem, amelyek elengedhetetlenek minden programtervező informatikus számára.

Hosszútávó céljaim az alkalmazással számosak, szeretném továbbfejleszteni közösségi open-source projektként

\cleardoublepage

% Függelékek (opcionális) - hosszabb részletező táblázatok, sok és/vagy nagy kép esetén hasznos
% Appendices (optional) - useful for detailed information in long tables, many and/or large figures, etc.
\appendix
\chapter{A továbbfejlesztési lehetőségekről}
\label{appx:further_development}

Lorem ipsum dolor sit amet, consectetur adipiscing elit. Pellentesque facilisis in nibh auctor molestie. Donec porta tortor mauris. Cras in lacus in purus ultricies blandit. Proin dolor erat, pulvinar posuere orci ac, eleifend ultrices libero. Donec elementum et elit a ullamcorper. Nunc tincidunt, lorem et consectetur tincidunt, ante sapien scelerisque neque, eu bibendum felis augue non est. Maecenas nibh arcu, ultrices et libero id, egestas tempus mauris. Etiam iaculis dui nec augue venenatis, fermentum posuere justo congue. Nullam sit amet porttitor sem, at porttitor augue. Proin bibendum justo at ornare efficitur. Donec tempor turpis ligula, vitae viverra felis finibus eu. Curabitur sed libero ac urna condimentum gravida. Donec tincidunt neque sit amet neque luctus auctor vel eget tortor. Integer dignissim, urna ut lobortis volutpat, justo nunc convallis diam, sit amet vulputate erat eros eu velit. Mauris porttitor dictum ante, commodo facilisis ex suscipit sed.

Sed egestas dapibus nisl, vitae fringilla justo. Donec eget condimentum lectus, molestie mattis nunc. Nulla ac faucibus dui. Nullam a congue erat. Ut accumsan sed sapien quis porttitor. Ut pellentesque, est ac posuere pulvinar, tortor mauris fermentum nulla, sit amet fringilla sapien sapien quis velit. Integer accumsan placerat lorem, eu aliquam urna consectetur eget. In ligula orci, dignissim sed consequat ac, porta at metus. Phasellus ipsum tellus, molestie ut lacus tempus, rutrum convallis elit. Suspendisse arcu orci, luctus vitae ultricies quis, bibendum sed elit. Vivamus at sem maximus leo placerat gravida semper vel mi. Etiam hendrerit sed massa ut lacinia. Morbi varius libero odio, sit amet auctor nunc interdum sit amet.

Aenean non mauris accumsan, rutrum nisi non, porttitor enim. Maecenas vel tortor ex. Proin vulputate tellus luctus egestas fermentum. In nec lobortis risus, sit amet tincidunt purus. Nam id turpis venenatis, vehicula nisl sed, ultricies nibh. Suspendisse in libero nec nisi tempor vestibulum. Integer eu dui congue enim venenatis lobortis. Donec sed elementum nunc. Nulla facilisi. Maecenas cursus id lorem et finibus. Sed fermentum molestie erat, nec tempor lorem facilisis cursus. In vel nulla id orci fringilla facilisis. Cras non bibendum odio, ac vestibulum ex. Donec turpis urna, tincidunt ut mi eu, finibus facilisis lorem. Praesent posuere nisl nec dui accumsan, sed interdum odio malesuada.
\cleardoublepage

% Irodalomjegyzék (kötelező)
% Bibliography (mandatory)
\addcontentsline{toc}{chapter}{\biblabel}
\printbibliography[title=\biblabel]
\cleardoublepage

% Ábrajegyzék (opcionális) - 3-5 ábra fölött érdemes
% List of figures (optional) - useful over 3-5 figures
\addcontentsline{toc}{chapter}{\lstfigurelabel}
\listoffigures
\cleardoublepage

% Táblázatjegyzék (opcionális) - 3-5 táblázat fölött érdemes
% List of tables (optional) - useful over 3-5 tables
\addcontentsline{toc}{chapter}{\lsttablelabel}
\listoftables
\cleardoublepage

% Forráskódjegyzék (opcionális) - 3-5 kódpélda fölött érdemes
% List of codes (optional) - useful over 3-5 code samples
\addcontentsline{toc}{chapter}{\lstcodelabel}
\lstlistoflistings
\cleardoublepage

% Jelölésjegyzék (opcionális)
% List of symbols (optional)
%\printnomenclature

\end{document}
