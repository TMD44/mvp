\chapter{Továbbfejlesztési lehetőségek}
\label{appx:further_development}

Egy alkalmazás teljes egészében soha nem készül el, mindig lesz rajta mit javítani, mindig lesznek potenciális új funkciók, továbbfejlesztési lehetőségek, ez alól természetesen a Multimédia Vizualizációs Platform sem kivétel.

Az alábbiakban listázom - a teljesség igénye nélkül - a főbb továbbfejlesztési lehetőségeket, ötleteimet:
\begin{itemize}
    \item {\textbf {Internationalization (i18n): }} Egy olyan funkció implementálása, amely megírása után csak különböző nyelvek fájl megadása szükséges. Tehát például az angol alapértelmezett mellett, egy {\it hu.json} vagy {\it de.json} fájl, amelyet az alkalmazás felismer és rögtön be is húz.
	\item {\textbf {Több téma: }} Minden komolyan vehető alkalmazás több témát ajánl felhasználó számára, de legalábbis egy ``dark'' és ``light'' kettőst. Ennek a funkciónak az alapjai már el vannak vetve, tekintve Sass-t használtam amely ehhez kiválóan alkalmazható.
	\item {\textbf {Drag and drop: }} A média kártyák még felhasználó barátiságát erősítené a Drag and drop funkció, lejátszási listákhoz lehetne hozzáadni ha ráhúzzuk a listára, sorrendet lehetne átrendezni stb.
	\item {\textbf {Szerkesztő: }} A média tartalmakat jelenleg a felhasználói felületről nem lehet kézzel szerkeszteni, ez is egy igen kényelmes megoldást jelentene a felhasználók számára.
\end{itemize}
