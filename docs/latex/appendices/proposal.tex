% Témabejelentő
% Thesis Proposal

\thispagestyle{empty}
{\bf \huge {Szakdolgozati témabejelentő}}

\begin{flushleft}
    {\bf {Hallgató adatai:}}\\
    \> \> \> \> {\emph {Név: }} \authorname\\
    \> \> \> \> {\emph {Neptun kód: }} \neptuncode\\
    \vspace{0.3cm}

    {\bf {Képzési adatok:}}\\
    \> \> \> \> {\emph {Szak: }} \degreename\\
    \> \> \> \> {\emph {Tagozat: }} \studytype\\
    \vspace{0.3cm}

    {\bf {Témavezető neve:}} \supname\\
    \> \> \> \> {\emph {Munkahelyének neve, tanszéke: }}\\
    \> \> \> \> \> \> \> \> {\small {ELTE Informatikai Kar, Adattudományi és Adattechnológiai Tanszék}}\\
    \> \> \> \> {\emph {Munkahelyének címe: }}\\
    \> \> \> \> \> \> \> \> {\small {117 Budapest, Pázmány Péter sétány 1/C., 2. emelet, 2.420-as szoba}}\\
    \> \> \> \> {\emph {Beosztás és iskolai végzettsége: }} \\
    \> \> \> \> \> \> \> \> {\small \supaff}\\
\end{flushleft}

\vspace{0.5cm}

{\bf {A szakdolgozat címe:}} {Multimédia Vizualizációs Platform}\\
\vspace{1cm}
\> \> {\bf {A szakdolgozat témája:}}

A Multimédia Vizualizációs Platform egy számítógépre, külső HDD-re, SSD-re vagy Pendrive-ra letöltött multimédia - főleg filmek, sorozatok - vizualizációjára szolgál, tehát ezen tartalmak fogyasztására egy áttekinthető és könnyen kezelhető felületet biztosít. Hasonlót, amelyhez már hozzászokhattunk a különböző streaming szolgáltatóktól, csak mindezt nem egy külső adatbázisból betöltve, hanem a saját offline tartalmainkra építve. A felület biztosítja az alapvető funkcionalitásokat amelyeket egy videó lejátszó platformtól el lehet várni: filmek listázása, keresés, lejátszás, megállítás, előre-hátra pörgetés, nyelv és felirat választás, lejátszási listák létrehozása stb. Külön kezeli az egyrészes filmek, a több részes filmsorozatok és a több részes TV sorozatok felületeit. A filmekhez és sorozatokhoz továbbá meta-adatokat (Cím, Szereplők, Leírás, Műfajok stb.) tölt le/be, ezzel is növelve a felhasználói élményt.

\vspace{1cm}

{\it {Budapest}, {2020.12.01.}}
